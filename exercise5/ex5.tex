%% Dokumentenklasse (Koma Script) -----------------------------------------
\documentclass[%
   11pt,              % Schriftgroesse
   ngerman,           % wird an andere Pakete weitergereicht
   a4paper,           % Seitengroesse
   DIV11,             % Textbereichsgroesse (siehe Koma Skript Dokumentation !)
]{scrartcl}%     Klassen: scrartcl, scrreprt, scrbook
% -------------------------------------------------------------------------

\usepackage[utf8]{inputenc} % Font Encoding, benoetigt fuer Umlaute
\usepackage[ngerman]{babel}   % Spracheinstellung

\usepackage[T1]{fontenc} % T1 Schrift Encoding
\usepackage{textcomp}    % Zusatzliche Symbole (Text Companion font extension)
\usepackage{lmodern}     % Latin Modern Schrift
\usepackage{listings}
\usepackage{framed}
\usepackage{amssymb}
\usepackage{amsmath}
\usepackage{framed}
\usepackage{listings}
%für die Konfusionsmatrizen
\usepackage{csvsimple}
\usepackage{lscape}


\usepackage[left=2cm,right=3cm,top=2cm,bottom=2cm,includeheadfoot]{geometry}
%Kopf- und Fußzeile
\usepackage{fancyhdr}
%Grafiken einbetten
\usepackage{graphicx}

\pagestyle{fancy}
\fancyhf{}
%Übungsteilnehmer
\fancyhead[L]{Matthias Hansen, 331600~~Lukas Huwald, 322890\\}
%Kopfzeile mittig
\fancyhead[R]{NLP Exercise05}
%Linie oben
\renewcommand{\headrulewidth}{0.5pt}

\setlength{\parskip}{1ex}

%Fußzeile links bzw. innen
\fancyfoot[L]{}
%Fußzeile rechts bzw. außen
\fancyfoot[R]{\thepage}
%Linie unten
\renewcommand{\footrulewidth}{0.5pt}
%% Dokument Beginn %%%%%%%%%%%%%%%%%%%%%%%%%%%%%%%%%%%%%%%%%%%%%%%%%%%%%%%%
\begin{document}
\section*{Task 1}
We start with the upper bound on $b$. We have
\begin{equation*}
	\frac{n_1}{b} - \frac{2n_2}{1-b} = \sum_{r \geq 3}\frac{r n_r}{r - 1 - b} \geq 0
\end{equation*}
It follows that
\begin{align*}
	\frac{n_1}{b} &\geq \frac{2 n_2}{1-b} &\Rightarrow \\
	n_1 &\geq b\cdot\frac{2n_2}{1-b} &\Rightarrow \\
	n_1(1-b) &\geq b \cdot 2 n_2 &\Rightarrow \\
	n_1 - b \cdot n_1 &\geq b \cdot 2 n_2 &\Rightarrow \\
	n_1 &\geq b(2 n_2 + n_1) &\Rightarrow \\
	b &\leq \frac{n_1}{n_1 + 2 n_2}
\end{align*}
Next up is the lower bound on $b$. We have
\begin{equation*}
	\frac{n_1}{b} = \frac{2 n_2}{1-b} + \sum_{r \geq 3}\frac{r n_r}{r - 1 - b} \leq \frac{2 n_2}{1-b} + \sum_{r \geq 3}r n_r
\end{equation*}
We now apply the upper bound on the $1-b$ in the denominator on the right side and obtain
\begin{equation}\label{e1}
	\frac{n_1}{b} \leq \frac{2 n_2}{1 - \frac{n_1}{n_1 + 2n_2}} + \sum_{r \geq 3}r n_r
\end{equation}
Some simplifications show
\begin{equation*}
	\frac{2 n_2}{1 - \frac{n_1}{n_1 + 2n_2}} = \frac{2n_2}{\frac{n_1+2n_2-n_1}{n_1+2n_2}} = \frac{2n_2}{\frac{2n_2}{n_1+2n_2}} = 2n_2 \cdot \frac{n_1+2n_2}{2n_2} = n_1 + 2n_2
\end{equation*}
Putting this into the above inequation (\ref{e1}) leads to
\begin{equation*}
	\frac{n_1}{b} \leq n_1 + 2n_2 + \sum_{r \geq 3}r n_r = N
\end{equation*}
So we have
\begin{equation*}
	\frac{n_1}{N} \leq b
\end{equation*}
Finally we derive the more complex lower bound on $b$. Again starting with
\begin{equation*}
	\frac{n_1}{b} - \frac{2n_2}{1-b} = \sum_{r \geq 3}\frac{r n_r}{r - 1 - b} 
\end{equation*}
we multiply with $(1-b)$ and get
\begin{align*}
	\frac{n_1}{b} - n_1 - 2n_2 &= \sum_{r \geq 3}\frac{1-b}{r-1-b}rn_r &\Rightarrow \\
	b &= \frac{n_1}{n_1 + 2n_2 + \sum_{r \geq 3}\frac{1-b}{r-1-b}rn_r}
\end{align*}
So to derive the bound
\begin{equation*}
	b \geq \frac{n_1}{n_1 + 2n_2 + \frac{1}{2}(N - n_1 - 2n_2)}
\end{equation*}
It remains to show that
\begin{equation}\label{e2}
	\sum_{r \geq 3}\frac{1-b}{r-1-b}rn_r \leq \frac{1}{2}(N - n_1 - 2n_2)
\end{equation}
We first observe that for $r\geq3$
\begin{equation*}
	\frac{1-b}{r-1-b} \leq \frac{1-b}{r-1-(r-1)b} = \frac{1-b}{(r-1)(1-b)} = \frac{1}{r-1}
\end{equation*}
Putting this into the left hand side of (\ref{e2}), we get
\begin{equation*}
	\sum_{r \geq 3}\frac{1-b}{r-1-b}rn_r \leq \sum_{r \geq 3}\frac{1}{r-1}rn_r \underset{(r \geq 3)}{\leq} \sum_{r \geq 3}\frac{1}{2}rn_r = \frac{1}{2}\sum_{r \geq 3}rn_r = \frac{1}{2}(N - n_1 - 2n_2)
\end{equation*}
which proves (\ref{e2}) and thus the bound.
\section*{Task 2}

\end{document}