%% Dokumentenklasse (Koma Script) -----------------------------------------
\documentclass[%
   11pt,              % Schriftgroesse
   ngerman,           % wird an andere Pakete weitergereicht
   a4paper,           % Seitengroesse
   DIV11,             % Textbereichsgroesse (siehe Koma Skript Dokumentation !)
]{scrartcl}%     Klassen: scrartcl, scrreprt, scrbook
% -------------------------------------------------------------------------

\usepackage[utf8]{inputenc} % Font Encoding, benoetigt fuer Umlaute
\usepackage[ngerman]{babel}   % Spracheinstellung

\usepackage[T1]{fontenc} % T1 Schrift Encoding
\usepackage{textcomp}    % Zusatzliche Symbole (Text Companion font extension)
\usepackage{lmodern}     % Latin Modern Schrift
\usepackage{listings}
\usepackage{framed}
\usepackage{amssymb}
\usepackage{amsmath}
\usepackage{framed}
\usepackage{listings}
%für die Konfusionsmatrizen
\usepackage{csvsimple}
\usepackage{lscape}


\usepackage[left=2cm,right=3cm,top=2cm,bottom=2cm,includeheadfoot]{geometry}
%Kopf- und Fußzeile
\usepackage{fancyhdr}
%Grafiken einbetten
\usepackage{graphicx}

\pagestyle{fancy}
\fancyhf{}
%Übungsteilnehmer
\fancyhead[L]{Matthias Hansen, 331600~~Lukas Huwald, 322890\\}
%Kopfzeile mittig
\fancyhead[R]{NLP Exercise04}
%Linie oben
\renewcommand{\headrulewidth}{0.5pt}

\setlength{\parskip}{1ex}

%Fußzeile links bzw. innen
\fancyfoot[L]{}
%Fußzeile rechts bzw. außen
\fancyfoot[R]{\thepage}
%Linie unten
\renewcommand{\footrulewidth}{0.5pt}
%% Dokument Beginn %%%%%%%%%%%%%%%%%%%%%%%%%%%%%%%%%%%%%%%%%%%%%%%%%%%%%%%%
\begin{document}
\section*{Task 1}
\section*{Task 2}
For the Turing good estimates, we have
\begin{equation*}
	\sum_{h,w}p(h,w) = \sum_{r=0}^R p_r\cdot n_r
\end{equation*}
where $p_R$ is some value $0\leq p_R \leq 1$ estimated indepently and the leaving one out estimates for the other $p_r$ are
\begin{equation*}
	p_r = \frac{(1-n_R p_R) (r+1)n_{r+1}}{N\cdot n_r}
\end{equation*}
We find
\begin{align*}
	\sum_{r=0}^R p_r\cdot n_r &= n_R\cdot p_R + \sum_{r=0}^{R-1} \frac{(1-n_R p_R) (r+1)n_{r+1}}{N\cdot n_r} \cdot n_r \\
	&= n_R\cdot p_R + \sum_{r=0}^{R-1} \frac{(1-n_R p_R) (r+1)n_{r+1}}{N} \\
	&= n_R\cdot p_R + \frac{1-n_R p_R}{N} \cdot \sum_{r=0}^{R-1} (r+1)n_{r+1} \\
	&= n_R\cdot p_R + \frac{1-n_R p_R}{N} \cdot \sum_{r=1}^{R} r\cdot n_r \\
	&= n_R\cdot p_R + \frac{1-n_R p_R}{N} \cdot N \\
	&= n_R\cdot p_R + 1-n_R p_R\\
	&= 1
\end{align*}
\section*{Task 3}
Assume we have classes $c_1,\ldots,c_i,\ldots c_C$.
The formula is 
\begin{equation*}
	p(w|v) = \sum_{i=1}^C \sum_{j=1}^C p(w|c_i) \cdot p(c_i|c_j) \cdot p(c_j|v)
\end{equation*}
For the derivation, we use the standard notation $p(X=x)$ for the probability that random variable $X$ has value $x$, instead of the short form notation $p(x)$. We use the random variables $X_1$ for the word $w$ and $X_2$ for history $v$. We also use $C_1$ for the class of the word and $C_2$ for the class of the history. At the steps marked $(*)$ in the derivation we use the independence assumption from the task (the distributions do not depend on other variables).
\begin{align*}
	p(X_1 = w | X_2 = v) &= \sum_{i=1}^C p(X_1 = w, C_1 = c_i | X_2 = v) \\
	&= \sum_{i=1}^C p(X_1 = w|C_1 = c_i, X_2 = v) \cdot p(C_1 = c_i| X_2 = v) \\
	&\overset{(*)}{=} \sum_{i=1}^C p(X_1 = w|C_1 = c_i) \cdot p(C_1 = c_i| X_2 = v) \\
	&= \sum_{i=1}^C \sum_{j=1}^C p(X_1 = w|C_1 = c_i) \cdot p(C_1 = c_i, C_2 = c_j| X_2 = v) \\
	&= \sum_{i=1}^C \sum_{j=1}^C p(X_1 = w|C_1 = c_i) \cdot p(C_1 = c_i| C_2 = c_j, X_2 = v) \cdot p(C_2 = c_j | X_2 = v) \\
	&\overset{(*)}{=} \sum_{i=1}^C \sum_{j=1}^C p(X_1 = w|C_1 = c_i) \cdot p(C_1 = c_i| C_2 = c_j) \cdot p(C_2 = c_j | X_2 = v)
\end{align*}
\section*{Task 4}
\end{document}