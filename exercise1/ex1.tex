%% Dokumentenklasse (Koma Script) -----------------------------------------
\documentclass[%
   11pt,              % Schriftgroesse
   ngerman,           % wird an andere Pakete weitergereicht
   a4paper,           % Seitengroesse
   DIV11,             % Textbereichsgroesse (siehe Koma Skript Dokumentation !)
]{scrartcl}%     Klassen: scrartcl, scrreprt, scrbook
% -------------------------------------------------------------------------

\usepackage[utf8]{inputenc} % Font Encoding, benoetigt fuer Umlaute
\usepackage[ngerman]{babel}   % Spracheinstellung

\usepackage[T1]{fontenc} % T1 Schrift Encoding
\usepackage{textcomp}    % Zusatzliche Symbole (Text Companion font extension)
\usepackage{lmodern}     % Latin Modern Schrift
\usepackage{listings}
\usepackage{framed}
\usepackage{amssymb}
\usepackage{amsmath}
\usepackage{framed}
\usepackage{listings}
\usepackage[left=2cm,right=3cm,top=2cm,bottom=2cm,includeheadfoot]{geometry}
%Kopf- und Fußzeile
\usepackage{fancyhdr}

\pagestyle{fancy}
\fancyhf{}
%Übungsteilnehmer
\fancyhead[L]{Matthias Hansen, XXXXXX~~Lukas Huwald, 322890\\}
%Kopfzeile mittig
\fancyhead[R]{NLP Exercise01}
%Linie oben
\renewcommand{\headrulewidth}{0.5pt}

%Fußzeile links bzw. innen
\fancyfoot[L]{}
%Fußzeile rechts bzw. außen
\fancyfoot[R]{\thepage}
%Linie unten
\renewcommand{\footrulewidth}{0.5pt}
%% Dokument Beginn %%%%%%%%%%%%%%%%%%%%%%%%%%%%%%%%%%%%%%%%%%%%%%%%%%%%%%%%
\begin{document}
\section*{Task 1}
\section*{Task 2}
\begin{equation*}
	p(h|e_1,e_2) = \frac{p(e_1,e_2|h) \cdot p(h)}{p(e_1,e_2)} = \frac{p(e_1|e_2,h) \cdot p(e_2|h) \cdot p(h)}{p(e_1,e_2)}
\end{equation*}
\subsection*{a)} From the above formula we see that the target probability can be calculated using the values of (ii). The values of (i) are not sufficient as we can not determine $p(e_1,e_2|h)$ without the conditional independence assumption. The values of (iii) are not sufficient because we can not determine $p(e_1,e_2)$ in the denominator.
\subsection*{b)} The values of (ii) are of course still sufficient. Using the conditional independence assumption, the right hand side of the above equation becomes
\begin{equation*}
\frac{p(e_1|h) \cdot p(e_2|h) \cdot p(h)}{p(e_1,e_2)}
\end{equation*}
and can be calculated given the values from (i). The values from (iii) are still not sufficient, for the same reason as in a).
\section*{Task 3}
Let $X, Y$ be independent poisson distributed random variables with parameters $\lambda_X, \lambda_Y$ and distributions $p_{\lambda_X}, p_{\lambda_Y}$. \\
Because $X$ and $Y$ are independent, the probability distribution of $X+Y$ is
\begin{equation*}
	p_\text{Sum}(n) = \sum_{i+j=n} p_{\lambda_X}(i) \cdot p_{\lambda_Y}(j), \qquad n \in \mathbb{N}_0.
\end{equation*} 
We obtain
\begin{align*}
	p_\text{Sum}(n) &= \sum_{i+j=n} p_{\lambda_X}(i) \cdot p_{\lambda_Y}(j) \\
	&= \sum_{i=0}^n p_{\lambda_X}(i) \cdot p_{\lambda_Y}(n-i) \\
	&= \sum_{i=0}^n \frac{e^{-\lambda_X}\lambda_X^i}{i!} \frac{e^{-\lambda_Y}\lambda_Y^{n-i}}{(n-i)!} \\
	&= e^{-(\lambda_X + \lambda_Y)} \cdot \sum_{i=0}^n  \frac{\lambda_X^i\lambda_Y^{n-i}}{i!(n-i)!} \\
	&= \frac{e^{-(\lambda_X + \lambda_Y)}}{n!} \cdot \sum_{i=0}^n  {n \choose i} \lambda_X^i\lambda_Y^{n-i} \\
	&\underset{\text{Binomial Theorem}}{=} \frac{e^{-(\lambda_X + \lambda_Y)}}{n!} \cdot (\lambda_X + \lambda_Y)^n \\
\end{align*}
This is the poisson distribution with parameter $\lambda_X + \lambda_Y$, so the sum $X + Y$ is also poisson distributed.
\section*{Task 4}
\end{document}
